\documentclass[12pt]{article}
\usepackage[spanish,activeacute]{babel}
\usepackage[utf8]{inputenc}
\usepackage[spanish]{babel}
\spanishdecimal{.}
\usepackage{amsmath}
\usepackage{amsfonts}
\usepackage{amssymb}
\usepackage{graphicx}
\topmargin -2.5 cm
\oddsidemargin -1.5 cm
\textheight 23.5 cm
\textwidth 18.5cm

\title{Problema Resuelto}
\author{Identificación: 6666}
\begin{document}
\maketitle


El texto está disponible bajo la Licencia Creative Commons Atribución Compartir Igual 3.0
\begin{enumerate} 

\item Enunciado del problema. No use valores numéricos para las variables del problema

  \begin{minipage}{0.4\linewidth}
    \includegraphics[scale=0.95]{bloques}
  \end{minipage}
  \begin{minipage}{0.6\linewidth}
    \begin{enumerate}
    \item Primera pregunta. Use sus iniciales para identificar el  siguiente ``label'':
      \label{item:JVa}
    \item Segunda pregunta
      \label{item:JVb}
    \item Evalúe sus respuestas para valores $m=3\ $Kg, etc
      \label{item:JVc}
    \end{enumerate}
  \end{minipage}
  
  \textbf{Solución}
  \begin{itemize}
  \item[\ref{item:JVa})] Solución primera pregunta ...
  \item[\ref{item:JVb})] Solución segunda pregunta ...
  \item[\ref{item:JVc})] Solución tercera pregunta ...
  \end{itemize}

\end{enumerate}
\end{document}
